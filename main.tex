\documentclass[a4paper]{article}

\input{./preambles/preamble}
\usetikzlibrary{external,
angles,
decorations.pathmorphing,
calc,intersections,through,backgrounds,
decorations.pathreplacing,
patterns,shadings,
arrows,
shapes.geometric}
% save compiled tikz plots; enable --shell-escape
\tikzexternalize[prefix=./tikz/]
% Sebastian did it
\usepackage[outline]{contour}
\contourlength{2pt}

\input{./preambles/unicode}

\setdefaultlanguage{english}
\setotherlanguages{german,french,italian,greek,latin,russian}
\setmainfont{CMU Serif}
\setsansfont{CMU Sans Serif}
% \newfontfamily{\greekfont}{CMU Serif}
% \newfontfamily{\greekfontsf}{CMU Sans Serif}

\input{./preambles/math-single}
\input{./preambles/math-brac}
\input{./preambles/math-thm}
\input{./preambles/phys-chem}

\newcommand{\RomaN}[1]{%
  \textup{\uppercase\expandafter{\romannumeral#1}}%
}

\usepackage[
	bibencoding  = auto,
	backend      = biber,
	sorting      = nyt,
	sortlocale   = auto,
	hyperref     = true,
	backref      = true,
	style        = phys,
	doi          = false,
	eprint       = true,
	isbn         = true,
	url          = true,
	subentry     = true,
	articletitle = true,
	biblabel     = brackets,
	chaptertitle = true,
	pageranges   = true,
	%refsegment   = section,
]{biblatex}
\addbibresource{main.bib}

%\input{./preambles/biblatex-user}

% \setcounter{secnumdepth}{6}
% So that paragraph is also numbered

\numberwithin{equation}{subsection}

\title{Wightman function of Klein--Fock--Gordon field}
\author{Yi-Fan Wang}


\begin{document}
\maketitle

%\begin{abstract}
%Your abstract.
%\end{abstract}

\tableofcontents

%1234567890123456789012345678901234567890123456789012345678901234567890123456789
\section{Covariant differential}

%1234567890123456789012345678901234567890123456789012345678901234567890123456789
$\phi$ a $\BbbC$-valued $0$-form
\begin{align}
\DD\phi \coloneqq \rbr{\dd - \ii e A}\wedge \phi,
\qquad
\DD\phi^* \coloneqq \rbr{\dd + \ii e A}\wedge \phi^*,
\end{align}
$A$ the $\mfraku(1)$-valued connection form.

%1234567890123456789012345678901234567890123456789012345678901234567890123456789
\section{Covariant codifferential}

%1234567890123456789012345678901234567890123456789012345678901234567890123456789

Define the covariant codifferential of a $\BbbC$-valued $k$-form $\zeta$ as 
follows. Let $\eta$ be an arbitrary $\BbbC$-valued $(k-1)$-form.
\begin{align}
\int \dd\rbr{\eta^* \wedge \star \zeta} &\equiv
\int \dd \eta^* \wedge \star \zeta - (-)^{k} \eta^* \wedge \dd\star\zeta
\eqqcolon
\int \DD \eta^* \wedge \star \zeta - \eta^* \wedge \star\DD^{\dagger}\zeta
\\
&=
\int \DD \eta^* \wedge \star \zeta - \ii e A \wedge \eta^* \wedge \star \zeta -
	(-)^{k} \eta^* \wedge \dd\star\zeta
\nonumber \\
&=
\int \DD \eta^* \wedge \star \zeta +
	\eta^* \wedge (-)^{k} \ii e A \wedge \star \zeta -
	(-)^{k} \eta^* \wedge \dd\star\zeta
\nonumber \\
&=
\int \DD \eta^* \wedge \star \zeta - \eta^* \wedge 
	\star (-)^{k} \star^{-1} \rbr{\dd - \ii e A} \wedge \star \zeta\,.
\end{align}
\begin{empheq}[box=\fbox]{align}
\DD^{\dagger} \zeta = (-)^{k} \star^{-1}
	\rbr{\dd - \ii e A} \wedge \star \zeta\,.
\end{empheq}

%1234567890123456789012345678901234567890123456789012345678901234567890123456789
\section{Maxwell--Klein--Fock--Gordon theory}

%1234567890123456789012345678901234567890123456789012345678901234567890123456789
\begin{align}
S = \int -\DD\phi^*\wedge\star\DD\phi - m^2 \phi^*\wedge\star\phi
	-\frac{1}{2} F \wedge \star F.
\end{align}

\begin{align}
\dva \DD\phi = -\ii e \dva A\phi + \DD \dva \phi.
\end{align}

\begin{align}
\begin{split}
\dva\rbr{\DD \phi^*\wedge\star\DD\phi} &= 
\dd\rbr{\dva\phi^*\wedge\star\DD\phi + \dva\phi\wedge\star\DD\phi^*}
\\
&\quad\,+
\dva\phi^*\wedge\star\DD^{\dagger}\DD\phi +
\dva\phi\wedge\star\DD^{\dagger}\DD\phi^*
\\
&\quad\,+
\dva A \wedge\rbr{\ii e \rbr{\phi^* \star\DD\phi - \phi\star\DD\phi^*}}\,,
\end{split}
\\
\dva\rbr{F\wedge\star F} &=
	2 \dd\rbr{\dva A \wedge \star F} - 2 \dva A \wedge \dd {\star F}\,.
\end{align}

\begin{align}
\dva S &= \int -\dd\rbr{\dva\phi^*\wedge\star\DD\phi + \dva\phi\wedge\star\DD\phi^* + \dva A \wedge\star F}
\nonumber \\
&\quad\,+
\dva\phi^*\wedge\star\rbr{\DD^{\dagger}\DD-m^2}\phi +
\dva\phi\wedge\star\rbr{\DD^{\dagger}\DD-m^2}\phi^* +
\nonumber \\
&\quad\,+
\dva A\wedge\rbr{-\dd{\star F} +
	\ii e \rbr{\phi^* \star\DD\phi - \phi \star\DD\phi^*}}.
\end{align}

%1234567890123456789012345678901234567890123456789012345678901234567890123456789
\subsection{Lorenz gauge}

%1234567890123456789012345678901234567890123456789012345678901234567890123456789
\begin{align}
\square^2 \coloneqq \rbr{\dd + \dd^{\dagger}}^2 = \dd \dd^{\dagger} + \dd^{\dagger} \dd.
\end{align}
\begin{align}
\dd{\star F} = \dd\star \dd A = \star (-)^2 \star^{-1}\dd\star\dd A = \star \dd^{\dagger} \dd A = \star\rbr{\square^2 - \dd \dd^{\dagger}} A\,.
\end{align}
One would like to have $\dd \dd^{\dagger} A = 0$, or $\dd^{\dagger} A = \text{const}$. This would be fulfiled if
\begin{align}
\dd^{\dagger} A = 0\,,
\end{align}
which is the Lorenz gauge\cite{Lorenz1867,Bladel1991a,Bladel1991b}.

%1234567890123456789012345678901234567890123456789012345678901234567890123456789
\section{Free Klein--Fock--Gordon equation in $(d+1)$-dimensions}

%1234567890123456789012345678901234567890123456789012345678901234567890123456789
In the absence of external electromagnetic field and Cartesian coordinates,
\begin{align}
0 = \rbr{\DD^{\dagger} \DD - m^2}\phi =
	\rbr{-\partial_t^2+\partial_x^2 - m^2}\phi\,.
\end{align}

Linearly independent solutions are
\begin{align}
\rfun{\exp}{\pm\ii\rbr{-\omega_{\vec{k}} t + \vec{k}\cdot\vec{x}}}\,,
\qquad
\omega_{\vec{k}} = \sqrt{\vec{k}^2 + m^2}\,.
\end{align}

One uses the Noether charge to normalise, which reads
\begin{align}
\rbr{\phi_1, \phi_2} = \int \dd^d x\, j^{0}
\end{align}
gives
\begin{align}
\rfun{\phi_{+}}{\omega_{\vec{k}},\vec{k}; t, x} = 
\frac{1}{\sqrt{2\omega_{\vec{k}}}}
\rfun{\exp}{+\ii\rbr{-\omega_{\vec{k}} t + \vec{k}\cdot\vec{x}}}\,,
\qquad
\rfun{\phi_{-}}{\omega_{\vec{k}},\vec{k}; t, x} =
\frac{1}{\sqrt{2\omega_{\vec{k}}}}
\rfun{\exp}{-\ii\rbr{-\omega_{\vec{k}} t + \vec{k}\cdot\vec{x}}}\,,
\end{align}


%\input{sections/intro}
%\printbibliography[segment=1,heading=subbibintoc]


% Let's print the overall heading of the bibliography first:
%\printbibheading
%\printbibliography

\end{document}